\documentclass{beamer}
\usetheme{Warsaw}
\usecolortheme{default}
\usefonttheme{default}
\usepackage[T1]{fontenc}
\usepackage[utf8]{inputenc}
\usepackage{lmodern}
\usepackage{xcolor}
\usepackage{listings}
\lstloadlanguages{Ruby}
\lstset{%
basicstyle=\ttfamily\scriptsize,
commentstyle = \ttfamily\color{red},
keywordstyle=\ttfamily\color{blue},
stringstyle=\color{orange},
breaklines=true,
language=Ruby}

\title{MegaUltraTweet}
\author{Balz Aschwanden}
\date{\today}

\begin{document}
\frame{\titlepage} 

\section{Projektziel}
\begin{frame}
  \frametitle{Projektziel}
  Durch die Auswertung von Tweets \pause
  \begin{itemize}
    \item Trends sofort erfassen \pause
    \item Zugang zu bis jetzt unbekannten Quellen
  \end{itemize}
\end{frame}

\section{Data Mining}
\begin{frame}
  \frametitle{Data Mining} \pause
  \begin{itemize}
    \item Rollover alle 15 min \pause
    \item 400 Anfragen an Twitter \pause
    \item 400 Tweets pro Anfrage \pause
    \item 39 Startbegriffe \pause
    \item 20 häufigste Begriffe kommen dazu \pause
    \item 80 Starbegriffe als Obergrenze
  \end{itemize}
\end{frame}

\section{Information Ranking}
\begin{frame}
  \frametitle{Information Ranking} \pause
  \begin{itemize}
    \item Author \pause \textcolor{red}{Follower Count} \pause
    \item Tweet \pause \textcolor{red}{Retweet Count + Rank Author} \pause
    \item Hashtag \pause \textcolor{red}{Times Used} \pause
    \item Author/Hashtag \textcolor{red}{Times Used}
    \item Hashtag/Hashtag \textcolor{red}{Times Used}
  \end{itemize}
\end{frame}

\section{Trending}
\begin{frame}
  \frametitle{Trending: Definition} \pause
  \begin{block}{Trend}
    Der Unterschied in der \textbf{Menge an Verwendungen} (eines Begriffs) von einem Intervall zum nächsten. \\ \pause
    Die Art des Trends definiert die Länge des Intervalls. \pause
    \begin{itemize}
      \item Short Term: 15 min \pause
      \item Long Term: 48 h
    \end{itemize}
  \end{block}
\end{frame}

\begin{frame}
  \frametitle{Trending: Berechnung I} \pause
  \begin{center}
      Pro Rollover wird erfasst, wie oft ein Hashtag verwendet wird. \\ \pause
      Um den Trend zu berechnen, wird am Ende jedes Rollovers der Unterschied zwischen dem jetzigen und dem vorherigen Intervall erfasst.
  \end{center}
\end{frame}

\begin{frame}[fragile]
  \frametitle{Trending: Berechnung II} \pause
  Beispiel:
  \begin{itemize}
    \item Short Term: 15 min \pause \textcolor{red}{=> Interval = 1} \pause
    \item Array a = [15 20 10 4] \pause
  \end{itemize}
  
  \begin{lstlisting}
    current = self.times_used[0, interval]
  \end{lstlisting}\pause
  \textcolor{red}{=> [15]}\pause
  \begin{lstlisting}
    old = self.times_used[0+interval, interval]
  \end{lstlisting}\pause
  \textcolor{red}{=> [20]}\pause
  \begin{lstlisting}
    return current.inject(:+) - old.inject(:+)
  \end{lstlisting}\pause
  \textcolor{red}{=> 15-20 = -5}
\end{frame}

\section{Livedemo}
\begin{frame}
  \frametitle{This is it}
  \begin{center}
    % \href{http://pse1.iam.unibe.ch/}{\beamergotobutton{Live Demo MegaUltraTweet}} \\
    \href{http://localhost:3000/}{\beamergotobutton{Live Demo MegaUltraTweet}} \\
    \includegraphics[height=.4\textheight]{logo_bird}
  \end{center}
\end{frame}

\end{document}
