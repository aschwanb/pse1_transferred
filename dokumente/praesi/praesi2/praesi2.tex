\documentclass{beamer}
\usetheme{AnnArbor}
\usecolortheme{beaver}
\usefonttheme{default}
\usepackage[T1]{fontenc}
\usepackage[utf8]{inputenc}
\usepackage{lmodern}
\usepackage{hyperref}
\usepackage{xcolor}
\usepackage{listings}
\lstloadlanguages{Ruby}
\lstset{%
basicstyle=\ttfamily\scriptsize,
commentstyle = \ttfamily\color{red},
keywordstyle=\ttfamily\color{blue},
stringstyle=\color{orange},
breaklines=true,
language=Ruby}

\title{MegaUltraTweet}
\author{Balz Aschwanden}
\date{\today}

\begin{document}
\frame{\titlepage}

\section{Tech Review}
\begin{frame}
  \frametitle{Themen} \pause
  \begin{minipage}{.3\textwidth}
    \includegraphics[height=.3\textheight]{TwitterAPI}
  \end{minipage}%
  \hfill
  \begin{minipage}{.7\textwidth}
    \textbf{Twitter API}    
  \end{minipage}
\end{frame}

\begin{frame}
  \begin{center}
    \textbf{Wie nutze ich die Twitter API?}
  \end{center}
\end{frame}
\begin{frame}
  \frametitle{Registrierung bei Twitter}
    \begin{itemize}
    \item Twitter Account erstellen\pause
    \item Application bei Twitter registrieren\pause
    \item Access Tokens generieren
  \end{itemize}
\end{frame}

\begin{frame}[fragile]
  \frametitle{Twitter \& Ruby}
    \begin{lstlisting}
      # Initialize Twitter Client
      @client = Twitter::REST::Client.new do |config|
        config.consumer_key = Rails.application.secrets.twitter_client_consumer_key
        config.consumer_secret = Rails.application.secrets.twitter_client_consumer_secret
        config.access_token = Rails.application.secrets.twitter_client_access_token
        config.access_token_secret = Rails.application.secrets.twitter_client_access_token_secret
      end
    \end{lstlisting}    
\end{frame}
\begin{frame}
  \frametitle{Use Cases}\pause
  \begin{center}
    \textbf{Ruby Twitter Client Use Cases}\\
    \hyperlink{http://www.ibm.com/developerworks/library/os-dataminingrubytwitter}{IBM 2012}\pause
  \end{center}
\end{frame}

\begin{frame}
  \frametitle{Limitationen}
  \begin{itemize}
    \item Response format JSON
    \item 100 Tweets / User pro Request
    \item 450 Requests pro 15 min 
    \item \textit{dev.twitter.com} is your friend
  \end{itemize}
\end{frame}

\begin{frame}[fragile]
  \frametitle{Use Cases}
  \begin{lstlisting}
# Returns the top 10 trending topics for a specific WOEID, if trending information is available for it.
puts client.trends
\end{lstlisting}\pause
\begin{lstlisting}
# <Twitter::TrendResults:0x000000025409a0>
\end{lstlisting}\pause
\begin{lstlisting}
client.trends.each { |t| puts "#{t}" }
\end{lstlisting}\pause
\begin{lstlisting}
#<Twitter::Trend:0x00000001fad3f8>
#<Twitter::Trend:0x00000001fad3a8>
#<Twitter::Trend:0x00000001fad330>
#<Twitter::Trend:0x00000001fad2e0>
#<Twitter::Trend:0x00000001fad290>
#<Twitter::Trend:0x00000001fad218>
#<Twitter::Trend:0x00000001fad100>
#<Twitter::Trend:0x00000001fad0d8>
#<Twitter::Trend:0x00000001fad0b0>
#<Twitter::Trend:0x00000001facfc0>
\end{lstlisting}
\end{frame}

\begin{frame}[fragile]
  \frametitle{Use Cases}
  \begin{itemize}
    \item events
    \item name
    \item promoted\_content
    \item query
    \item url
  \end{itemize}\pause
  \begin{lstlisting}
client.trends.each { |t| puts "#{t.name}" }   
  \end{lstlisting}\pause
  \begin{lstlisting}
# ElClaciso
# BarcelonavsRealmadrid
# FCBReal
# Whatssap 539x647x1476
# La6Andalucia
# BeyazFutbolaDiyorumki
# Martin Silva
# Mathieu y Suarez
# BarcelonavsRealmadrid
# GrantsBedtimeStoriesTour
  \end{lstlisting}
\end{frame}

\begin{frame}[fragile]
  \frametitle{Use Cases}
  \begin{lstlisting}
    
  \end{lstlisting}
\end{frame}

\begin{frame}[fragile]
  \frametitle{Use Cases}
  \begin{lstlisting}
    
  \end{lstlisting}
\end{frame}



% Usecase: Suche nach Usern
% Anatomie des Users auflisten
% Usecase: Suche nach #Technology
% Wer schreibt die Tweets? (Usernamen)
% Wo werden die Tweets geschrieben? (Geolocation)
% Wann werden die Tweets geschrieben?
% Auflistung als Graph?
% 
% Limitierung der Api (Begrenzte Abfragen)
% 


\begin{frame}
  \begin{center}
      \textbf{Fragen ?}
  \end{center}
\end{frame}

\begin{frame}
  \begin{center}
      \textbf{Vielen Dank für die Aufmerksamkeit} \\
      \includegraphics[width=.4\textwidth]{megaultra_ruby}
  \end{center}
\end{frame}


\end{document}
