\documentclass{beamer}
\usetheme{AnnArbor}
\usecolortheme{beaver}
\usefonttheme{default}
\usepackage[T1]{fontenc}
\usepackage[utf8]{inputenc}
\usepackage{lmodern}
\usepackage{hyperref}
\usepackage{xcolor}
\usepackage{listings}
\lstloadlanguages{Ruby}
\lstset{%
basicstyle=\ttfamily\scriptsize,
commentstyle = \ttfamily\color{red},
keywordstyle=\ttfamily\color{blue},
stringstyle=\color{orange},
breaklines=true,
language=Ruby}

\title{MegaUltraTweet}
\author{Balz Aschwanden}
\date{\today}

\begin{document}
\frame{\titlepage}

\section{Tech Review}
\begin{frame}
  \frametitle{Twitter API}
  \begin{center}
      \includegraphics[height=.8\textheight]{TwitterAPI}
  \end{center}

\end{frame}

\begin{frame}
  \begin{center}
    \textbf{Wie nutze ich die Twitter API?}
  \end{center}
\end{frame}
\begin{frame}
  \frametitle{Admin Twitter}
  \begin{minipage}{.2\textwidth}
    \includegraphics[height=.2\textheight]{TwitterAPI}
  \end{minipage}%
  \hfill
  \begin{minipage}{.8\textwidth}
    Twitter Account erstellen \pause \\    
  \end{minipage}
  \begin{minipage}{.2\textwidth}
    \includegraphics[height=.2\textheight]{megaultra_ruby_notext}
  \end{minipage}%
  \hfill
  \begin{minipage}{.8\textwidth}
    Application bei Twitter registrieren \pause \\    
  \end{minipage}
  \begin{minipage}{.2\textwidth}
    \includegraphics[height=.2\textheight]{token}
  \end{minipage}%
  \hfill
  \begin{minipage}{.8\textwidth}
    Access Tokens generieren    
  \end{minipage}
\end{frame}

\begin{frame}[fragile]
  \frametitle{Twitter \& Ruby}
    \begin{lstlisting}
      require 'twitter'
      # Initialize Twitter Client
      @client = Twitter::REST::Client.new do |config|
        config.consumer_key = Rails.application.secrets.twitter_client_consumer_key
        config.consumer_secret = Rails.application.secrets.twitter_client_consumer_secret
        config.access_token = Rails.application.secrets.twitter_client_access_token
        config.access_token_secret = Rails.application.secrets.twitter_client_access_token_secret
      end
    \end{lstlisting}    
\end{frame}

\begin{frame}
  \frametitle{Limitationen}
  \begin{itemize}
    \item Response format JSON \pause
    \item Zeitfenster: 15 min\pause
    \item Anfragen pro Zeitfenster: 450\pause
    \item \textit{dev.twitter.com} is your friend
  \end{itemize}
\end{frame}

\begin{frame}
  \frametitle{Use Cases}
  \begin{center}
    \textbf{Ruby Twitter Client Use Cases}\\
    \hyperlink{http://www.ibm.com/developerworks/library/os-dataminingrubytwitter}{IBM 2012}
  \end{center}
\end{frame}

\begin{frame}[fragile]
  \frametitle{Use Cases}
  \begin{lstlisting}
# Returns the top 10 trending topics for a specific WOEID, if trending information is available for it.
puts client.trends
\end{lstlisting}\pause
\begin{lstlisting}
# <Twitter::TrendResults:0x000000025409a0>
\end{lstlisting}\pause
\begin{lstlisting}
client.trends.each { |t| puts "#{t}" }
\end{lstlisting}\pause
\begin{lstlisting}
#<Twitter::Trend:0x00000001fad3f8>
#<Twitter::Trend:0x00000001fad3a8>
#<Twitter::Trend:0x00000001fad330>
#<Twitter::Trend:0x00000001fad2e0>
#<Twitter::Trend:0x00000001fad290>
#<Twitter::Trend:0x00000001fad218>
#<Twitter::Trend:0x00000001fad100>
#<Twitter::Trend:0x00000001fad0d8>
#<Twitter::Trend:0x00000001fad0b0>
#<Twitter::Trend:0x00000001facfc0>
\end{lstlisting}\pause
\textit{dev.twitter.com} is your friend
\end{frame}

\begin{frame}[fragile]
  \frametitle{Use Cases}
  \begin{itemize}
    \item events
    \item name
    \item promoted\_content
    \item query
    \item url
  \end{itemize}\pause
  \begin{lstlisting}
client.trends.each { |t| puts "#{t.name}" }   
  \end{lstlisting}\pause
  \begin{lstlisting}
# ElClaciso
# BarcelonavsRealmadrid
# FCBReal
# Whatssap 539x647x1476
# La6Andalucia
# BeyazFutbolaDiyorumki
# Martin Silva
# Mathieu y Suarez
# BarcelonavsRealmadrid
# GrantsBedtimeStoriesTour
  \end{lstlisting}
\end{frame}

\begin{frame}[fragile]
  \frametitle{Use Cases}
  Tweets suchen\pause
  \begin{lstlisting}
tweets = client.search("#Technology OR #Technologie", :result_type => "recent").take(100).to_a
  \end{lstlisting}\pause
  Tweets nach Retweets sortieren und die populärsten drei ausgeben\pause
  \begin{lstlisting}
tweets.sort! { |a,b| a.retweet_count <=> b.retweet_count}
tweets.reverse!
tweets.first(3).each { |t| puts "Very popular tweet: #{t.text}" }    
  \end{lstlisting}
\end{frame}

\begin{frame}[fragile]
  \frametitle{Use Cases}
Wer steckt hinter diesen Tweets?\pause
  \begin{lstlisting}
tweets.first(3).each { |t| puts "Very popular tweet with #{t.retweet_count} retweets by #{t.user.name} from #{t.user.location}" }
# Very popular tweet with 23 retweets by Josefine Mahon from Lawrenceville, AL, US
# Very popular tweet with 3 retweets by Technews from Aperture Laboratories
# Very popular tweet with 3 retweets by Ankur from Ahmedabad 
  \end{lstlisting}
\end{frame}

\begin{frame}[fragile]
  \frametitle{Use Cases}
Wie bekannt sind diese Twitter Nutzer?\pause
  \begin{lstlisting}
tweets.first(3).each { |t| puts "User #{t.user.name} has #{t.user.followers_count} followers" }
# User Josefine Mahon has 63 followers
# User Technews has 395 followers
# User Ankur has 926 followers
  \end{lstlisting}
\end{frame}

\begin{frame}[fragile]
  \frametitle{Use Cases}
  Wo werden diese Tweets geschrieben?\pause
  \begin{lstlisting}
tweets = client.search("#Technology OR #Technologie", :result_type => "recent").take(100).to_a
tweetLocation = Hash.new
tweets.each do |t|
  if !tweetLocation[t.user.location.to_s].nil?
    tweetLocation[t.user.location.to_s] = tweetLocation[t.user.location.to_s] + 1
  else
    tweetLocation[t.user.location.to_s] = 1
  end
end
  \end{lstlisting}
\end{frame}


\begin{frame}[fragile]
  \frametitle{Use Cases}
  \begin{lstlisting}
tweetLocation = tweetLocation.sort_by { |key, value| value }
tweetLocation.reverse!
tweetLocation.first(5).each { |key, value| puts "#{key} => #{value}"}
  \end{lstlisting}\pause
  \begin{lstlisting}
#  => 26
# North Pole => 11
# New York => 3
# UK => 2
# Philippines => 2    
  \end{lstlisting}
\end{frame}

\begin{frame}[fragile]
  \frametitle{Use Cases}
  Was für Hashtags werden gleichzeitig mit \#technology eingesetzt?\pause
  \begin{lstlisting}
twitterClient = TwitterClient.new
twitterClient.search("#technology", 500)
tags = twitterClient.getHashtagsAsHash
tags.delete("#technology")
tags.first(5).each { |key, value| puts "#{key} => #{value}" }    
  \end{lstlisting}\pause
  \begin{lstlisting}
#tech => 44
#partner => 21
#mobilephones => 20
#startups => 18
#smartphones => 15    
  \end{lstlisting}
\end{frame}

\begin{frame}
  \begin{center}
      \textbf{Fragen ?}
  \end{center}
\end{frame}

\begin{frame}
  \begin{center}
      \textbf{Vielen Dank für die Aufmerksamkeit} \\
      \includegraphics[width=.4\textwidth]{megaultra_ruby}
  \end{center}
\end{frame}


\end{document}
