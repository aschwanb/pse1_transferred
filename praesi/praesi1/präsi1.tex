\documentclass{beamer}
\usetheme{AnnArbor}
\usecolortheme{beaver}
\usefonttheme{default}
\usepackage[T1]{fontenc}
\usepackage[utf8]{inputenc}
\usepackage{lmodern}

\title{MegaUltraTweet}
\author{Balz Aschwanden}
\date{\today}

\begin{document}
\frame{\titlepage} 

\section{Team}
\begin{frame}
  \frametitle{Team}
  \begin{minipage}{.2\textwidth}
    \includegraphics[height=.2\textheight]{balz}
  \end{minipage}%
  \hfill
  \begin{minipage}{.8\textwidth}
    \textbf{Key Account Manager} \\ Raoul Grossenbacher \pause \\    
  \end{minipage}
  \begin{minipage}{.2\textwidth}
    \includegraphics[height=.2\textheight]{balz}
  \end{minipage}%
  \hfill
  \begin{minipage}{.8\textwidth}
    \textbf{Chief Deliverable Officer} \\ Balz Aschwanden \pause \\    
  \end{minipage}
  \begin{minipage}{.2\textwidth}
    \includegraphics[height=.2\textheight]{balz}
  \end{minipage}%
  \hfill
  \begin{minipage}{.8\textwidth}
    \textbf{Master Tracker} \\ Simon Curty \pause \\    
  \end{minipage}
  \begin{minipage}{.2\textwidth}
    \includegraphics[height=.2\textheight]{balz}
  \end{minipage}%
  \hfill
  \begin{minipage}{.8\textwidth}
    \textbf{Quality Evangelist} \\ Daniel Ziltener \\    
  \end{minipage}
\end{frame}

\section{Problem}
\begin{frame}
  \frametitle{Problem}
  Wie bleibe ich über die neusten Entwicklungen in einem Feld informiert? \pause
    \begin{itemize}
       \item Blogs \pause \textcolor{red}{Nicht immer relevant} \pause
       \item Nachrichtenmagazine \pause \textcolor{red}{Zu viele Neuigkeiten} \pause
       \item Netzwerk \pause \textcolor{red}{Auf persönlichen Kontakt angewiesen}
    \end{itemize}       
\end{frame}  

\begin{frame}
  \frametitle{Folge}
  \begin{center}
    An wenige Quellen gebunden \pause\\
    % Ich will nicht täglich 100 Sachen in meinem RSS Feed
    Wirklich neue Entwicklungen gehen oft unter \pause\\
    % Wir bekommen sie nicht mit, weil wir zusehr damit beschäftigt sind, auf dem Laufenden zu bleiben
  \includegraphics[height=.5\textheight]{sadcrying}  
  \end{center}
\end{frame}
   
\section{Lösung}
\begin{frame}
  \frametitle{Lösung}
  \begin{center}
    Grundlegend Neues sofort erfassen \pause \\
    Zugang zu bis jetzt unbekannten Quellen \pause \\
    \includegraphics[height=.4\textheight]{happy}     
  \end{center}
\end{frame}

\begin{frame}
  \frametitle{Vorgehen}
  \begin{itemize}
    \item Auswertung des Twitter Streams
    \item Präsentation neuer Inhalte
    \item User erhält:
    \begin{itemize}
      \item Direkten Zugriff auf Quellen
      % Nur auf den relevanten von 50 Blogeinträgen
      \item Einfache Navigation im Begriffsfeld
      \item Gezielte Auswahl nach Schlagwörtern
    \end{itemize}
  \end{itemize}

\end{frame}

\section{Technik}
\begin{frame}
  \frametitle{Technik}
  \begin{minipage}{.3\textwidth}
    \includegraphics[height=.3\textheight]{RubyOnRails}
  \end{minipage}%
  \hfill
  \begin{minipage}{.7\textwidth}
    Ruby On Rails \pause \\    
  \end{minipage}
  \begin{minipage}{.3\textwidth}
    \includegraphics[height=.3\textheight]{MongoDB}
  \end{minipage}%
  \hfill
  \begin{minipage}{.7\textwidth}
    MongoDB \pause \\    
  \end{minipage}
  \begin{minipage}{.3\textwidth}
    \includegraphics[height=.3\textheight]{GitHub}
  \end{minipage}%
  \hfill
  \begin{minipage}{.7\textwidth}
    GitHub \\    
  \end{minipage}

\end{frame}

\section{Erste Iteration}
\begin{frame}
  \frametitle{Erste Iteration}
  \begin{itemize}
    \item Twitter-Login für den User \pause
    \item Verbindung Thema / Hashtag \pause
    \item Schwerpunkt Thema \textit{Technik} \pause
    \item Mobile first
  \end{itemize}
  
\end{frame}

\begin{frame}
  \frametitle{Namensgebung}
  \begin{center}
    Tweet bei Eintreten einer seltenen Kombination \pause \\
    \textbf{UltraTweet}
  \end{center}
\end{frame}

\begin{frame}
  \includegraphics[width=\textwidth]{BringItOnBro}
\end{frame}

\end{document}
